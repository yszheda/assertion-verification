%        File: manual.tex
%     Created: 二  2月 07 12:00 下午 2012 C
% Last Change: 二  2月 07 12:00 下午 2012 C
%
\documentclass[a4paper]{article}
\title{Manual}
\author{Shuai Yuan}
\date{}
\begin{document}
\maketitle{}

%%%%%%%%%%%%%%%%%%%%%%%%%%%%
% define new command
\newcommand{\bdm}
{
\begin{displaymath}
\begin{array}{rcl}
}
\newcommand{\edm}
{
\end{array}
\end{displaymath}
}
%%%%%%%%%%%%%%%%%%%%%%%%%%%%

The program ``translation'' parse SQL phrases into Why3ML program. It mainly contains the following parts:
\begin{itemize}
  \item[-] header of Why3ML program, mainly commands of importing libraries.
  \item[-] parser for the SQL table definition, this part translate the ``CREATE TABLE'' phrase into the definition of the type of the corresponding table's tuple.
  \item[-] parser for the SQL assertion, this part translate the ``CREATE ASSERTION'' phrase into ``predicate'' in the Why3ML program.
  \item[-] parser for the SQL INSERT command, this part translate the SQL INSERT command into a method in the Why3ML program.
  \item[-] parser for the SQL DELETE command, this part translate the SQL INSERT command into a method in the Why3ML program.
  \item[-] parser for the SQL UPDATE command, this part translate the SQL INSERT command into a method in the Why3ML program.
\end{itemize}

\section{Parser for the SQL Table Definition}
The source language of SQL table definition is expressed in the following grammar:
\bdm
\textrm{$<$table definition$>$} &::=& \textrm{CREATE TABLE \textit{$<$table name$>$}} \\ 
&& \left( \textrm{$<$table element list$>$} \right) \\
\textrm{$<$table element list$>$} &::=& \textrm{$<$table element$>$} \\
&|& \textrm{$<$table element list$>$} , \textrm{$<$table element$>$} \\
\textrm{$<$table element$>$} &::=& \textrm{\textit{$<$column name$>$} $<$data type$>$} \\
\textrm{$<$data type$>$} &::=& \textrm{INTEGER} \\
&& \textrm{SMALLINT} \\
&& \textrm{FLOAT} \\
&& \textrm{NUMERIC} \\
&& \textrm{BOOLEAN} 
% \\ && \textrm{CHAR}
\edm

\section{Parser for the SQL Assertion}
The grammar of SQL assertion is:
\bdm
\textrm{CREATE ASSERTION \textit{$<$assertion name$>$}}	\\
\textrm{CHECK $<$exists predicate$>$}
% \textrm{CHECK $<$search condition$>$}
\edm
\bdm
\textrm{$<$exists predicate$>$} &::=& \textrm{[ NOT ] EXISTS ( $<$query expression$>$ )} \\

\textrm{$<$query expression$>$} &::=& \textrm{SELECT *}	\\
% && \textrm{FROM $R_{1}$ $x_{1}$,$\cdots$, $R_{n}$ $x_{n}$ }	\\
&& \textrm{FROM $<$table list$>$}	\\
&& \textrm{WHERE $<$search condition$>$} \\

\textrm{$<$table list$>$} &::=& \textrm{\textit{$<$table name$>$}} ~ \textrm{\textit{$<$tuple name$>$}} \\
&& \textrm{$<$table list$>$} , \textrm{\textit{$<$table name$>$}} ~ \textrm{\textit{$<$tuple name$>$}} \\

\textrm{$<$search condition$>$} &::=& \textrm{$<$boolean term$>$} \\
&|& \textrm{$<$search condition$>$ OR $<$boolean term$>$} \\

\textrm{$<$boolean term$>$} &::=& \textrm{$<$boolean factor$>$} \\
&|& \textrm{$<$boolean term$>$ AND $<$boolean factor$>$} \\

\textrm{$<$boolean factor$>$} &::=& \textrm{$<$predicate$>$} \\
&|& \textrm{[ NOT ] ( $<$search condition$>$ )} \\

\textrm{$<$predicate$>$} &::=& \textrm{$<$exists predicate$>$} \\
&|& \textrm{$<$comparison predicate$>$} \\
&|& \textrm{$<$between predicate$>$} \\
&|& \textrm{$<$in predicate$>$} \\
&|& \textrm{$<$null predicate$>$} \\

% \textrm{$<$exists predicate$>$} &::=& \textrm{[ NOT ] EXISTS ( $<$query expression$>$ )} \\
% 
% \textrm{$<$query expression$>$} &::=& \textrm{SELECT *}	\\
% % && \textrm{FROM $R_{1}$ $x_{1}$,$\cdots$, $R_{n}$ $x_{n}$ }	\\
% && \textrm{FROM $<$table list$>$}	\\
% && \textrm{WHERE $<$search condition$>$} \\
% 
% \textrm{$<$table list$>$} &::=& \textrm{$<$table name$>$} ~ \textrm{$<$tuple name$>$} \\
% && \textrm{$<$table list$>$} , \textrm{$<$table name$>$} ~ \textrm{$<$tuple name$>$} \\
% 
\textrm{$<$comparison predicate$>$} &::=& \textrm{$<$expression$_{1}>$} ~ \textrm{$<$comp op$>$} ~ \textrm{$<$expression$_{2}>$} \\
\textrm{$<$comp op$>$} &::=&  =	
~|~	<>
~|~	<
~|~	\leq
~|~	>
~|~	\geq \\

\textrm{$<$expression$>$} &::=& \textrm{$<$term$>$} \\
&|& \textrm{$<$expression$>$} ~ \left\{ + ~|~ - \right\} ~ \textrm{$<$term$>$} \\

\textrm{$<$term$>$} &::=& \textrm{$<$factor$>$} \\
&|&	 \textrm{$<$term$>$} ~ \left\{ * ~|~ / \right\} ~ \textrm{$<$factor$>$} \\
  
\textrm{$<$factor$>$} &::=& \left( \textrm{$<$expression$>$} \right) \\
&|& \left[ + ~|~ - \right] \textrm{\textit{$<$constant$>$}} \\
&|& \left[ + ~|~ - \right] \textrm{$<$column$>$} \\

\textrm{$<$column$>$} &::=& \textrm{\textit{$<$tuple name$>$}.\textit{$<$attribute name$>$}} \\

\textrm{$<$between predicate$>$} &::=& \mbox{$<$expression$>$ [ NOT ]} \\
&& \mbox{BETWEEN \textit{$<$constant$_{1}>$} AND \textit{$<$constant$_{2}>$}} \\

\textrm{$<$in predicate$>$}	&::=& \textrm{$<$expression$>$} \textrm{[ NOT ] IN ( $<$in value list$>$ )} \\
\textrm{$<$in value list$>$} &::=& \textrm{\textit{$<$constant$>$}} \\
&|& \textrm{$<$in value list$>$} , \textrm{\textit{$<$constant$>$}} \\

\textrm{$<$null predicate$>$} &::=& \textrm{$<$column$>$} \textrm{ IS [ NOT ] NULL} \\

\edm

\section{Parser for the SQL INSERT statement}
The grammar of SQL insert statement is:
\bdm
\textrm{$<$insert statement$>$} &::=& \textrm{INSERT INTO \textit{$<$table name$>$} VALUES ( $<$column value list$>$ )} \\
&|& \textrm{INSERT INTO \textit{$<$table name$>$} ( $<$column name list$>$ )} \\
&& \textrm{VALUES ( $<$column value list$>$ )} \\
\textrm{$<$column name list$>$} &::=& \textrm{\textit{$<$column name$>$}} \\
&|& \textrm{$<$column name list$>$} , \textrm{\textit{$<$column name$>$}} \\
\textrm{$<$column value list$>$} &::=& \textrm{$<$column value$>$} \\
&|& \textrm{$<$column value list$>$} , \textrm{\textit{$<$column value$>$}} \\

\edm

\section{Parser for the SQL DELETE statement}
The grammar of SQL delete statement is:
\bdm
\textrm{$<$delete statement$>$} &::=& \textrm{DELETE FROM \textit{$<$target table name$>$}} \\
&& \textrm{[ USING $<$table reference list$>$ ]} \\
&& \textrm{[ WHERE $<$search condition$>$ ]} \\
\textrm{$<$table reference list$>$} &::=& \textrm{\textit{$<$table name$>$}} \\
&|& \textrm{$<$table reference list$>$} , \textrm{\textit{$<$table name$>$}} \\

\textrm{$<$search condition$>$} &::=& \textrm{$<$boolean term$>$} \\
&|& \textrm{$<$search condition$>$ OR $<$boolean term$>$} \\

\textrm{$<$boolean term$>$} &::=& \textrm{$<$boolean factor$>$} \\
&|& \textrm{$<$boolean term$>$ AND $<$boolean factor$>$} \\

\textrm{$<$boolean factor$>$} &::=& \textrm{$<$predicate$>$} \\
&|& \textrm{[ NOT ] ( $<$search condition$>$ )} \\

\textrm{$<$predicate$>$} &::=& \textrm{$<$comparison predicate$>$} \\
&|& \textrm{$<$between predicate$>$} \\
&|& \textrm{$<$in predicate$>$} \\
&|& \textrm{$<$null predicate$>$} \\
\edm
The left parts are the same as those in the SQL assertion, so they are omitted in this manual. 

\section{Parser for the SQL UPDATE statement}
The grammar of SQL update statement is:
\bdm
\textrm{$<$update statement$>$} &::=& \textrm{UPDATE \textit{$<$target table name$>$}} \\
&& \textrm{SET $<$set clause list$>$} \\
&& \textrm{[ FROM $<$table reference list$>$ ]} \\
&& \textrm{[ WHERE $<$search condition$>$ ]} \\
\textrm{$<$set clause list$>$} &::=& \textrm{$<$set clause$>$} \\
&& \textrm{$<$set clause list$>$} , \textrm{$<$set clause$>$} \\
\textrm{$<$set clause$>$} &::=& \textrm{$<$set column$>$} = \textrm{\textit{$<$constant$>$}} \\
\textrm{$<$set column$>$} &::=& \textrm{\textit{$<$table name$>$}.\textit{$<$attribute name$>$}} \\
\edm
The left parts are the same as those in the SQL delete statement grammar, so they are omitted in this manual. 


\end{document}


