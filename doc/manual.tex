%        File: manual.tex
%     Created: 二  2月 07 12:00 下午 2012 C
% Last Change: 二  2月 07 12:00 下午 2012 C
%
\documentclass[a4paper]{article}
\title{Manual}
\author{Shuai Yuan}
\date{}
\usepackage{amsfonts}
\begin{document}
\maketitle{}

%%%%%%%%%%%%%%%%%%%%%%%%%%%%
% define new command
\newcommand{\bdm}
{
\begin{displaymath}
\begin{array}{rcl}
}
\newcommand{\edm}
{
\end{array}
\end{displaymath}
}
%%%%%%%%%%%%%%%%%%%%%%%%%%%%

The program ``translation'' parse SQL phrases into Why3ML program. It mainly contains the following parts:
\begin{itemize}
  \item[-] header of Why3ML program, mainly commands of importing modules.
  \item[-] parser for the SQL table definition, this part translate the ``CREATE TABLE'' phrase into the definition of the type of the corresponding table's tuple.
  \item[-] parser for the SQL assertion, this part translate the ``CREATE ASSERTION'' phrase into ``predicate'' in the Why3ML program.
  \item[-] parser for the SQL INSERT command, this part translate the SQL INSERT command into a method in the Why3ML program.
  \item[-] parser for the SQL DELETE command, this part translate the SQL INSERT command into a method in the Why3ML program.
  \item[-] parser for the SQL UPDATE command, this part translate the SQL INSERT command into a method in the Why3ML program.
\end{itemize}

\section{Parser for the SQL Table Definition}
The source language of SQL table definition is expressed in the following grammar:
\bdm
\textrm{$<$table definition$>$} &::=& \textrm{CREATE TABLE \textit{$<$table name$>$}} \\ 
&& \left( \textrm{$<$table element list$>$} \right) \\
\textrm{$<$table element list$>$} &::=& \textrm{$<$table element$>$} \\
&|& \textrm{$<$table element list$>$} , \textrm{$<$table element$>$} \\
\textrm{$<$table element$>$} &::=& \textrm{\textit{$<$column name$>$} $<$data type$>$} \\
\textrm{$<$data type$>$} &::=& \textrm{INTEGER} \\
&& \textrm{SMALLINT} \\
&& \textrm{FLOAT} \\
&& \textrm{NUMERIC} \\
&& \textrm{BOOLEAN} 
% \\ && \textrm{CHAR}
\edm

\section{Parser for the SQL Assertion}
The grammar of SQL assertion is:
\bdm
\textrm{CREATE ASSERTION \textit{$<$assertion name$>$}}	\\
\textrm{CHECK $<$exists predicate$>$}
% \textrm{CHECK $<$search condition$>$}
\edm
\bdm
\textrm{$<$exists predicate$>$} &::=& \textrm{[ NOT ] EXISTS ( $<$query expression$>$ )} \\

\textrm{$<$query expression$>$} &::=& \textrm{SELECT *}	\\
% && \textrm{FROM $R_{1}$ $x_{1}$,$\cdots$, $R_{n}$ $x_{n}$ }	\\
&& \textrm{FROM $<$table list$>$}	\\
&& \textrm{WHERE $<$search condition$>$} \\

\textrm{$<$table list$>$} &::=& \textrm{\textit{$<$table name$>$}} ~ \textrm{\textit{$<$tuple name$>$}} \\
&& \textrm{$<$table list$>$} , \textrm{\textit{$<$table name$>$}} ~ \textrm{\textit{$<$tuple name$>$}} \\

\textrm{$<$search condition$>$} &::=& \textrm{$<$boolean term$>$} \\
&|& \textrm{$<$search condition$>$ OR $<$boolean term$>$} \\

\textrm{$<$boolean term$>$} &::=& \textrm{$<$boolean factor$>$} \\
&|& \textrm{$<$boolean term$>$ AND $<$boolean factor$>$} \\

\textrm{$<$boolean factor$>$} &::=& \textrm{$<$predicate$>$} \\
&|& \textrm{[ NOT ] ( $<$search condition$>$ )} \\

\textrm{$<$predicate$>$} &::=& \textrm{$<$exists predicate$>$} \\
&|& \textrm{$<$comparison predicate$>$} \\
&|& \textrm{$<$between predicate$>$} \\
&|& \textrm{$<$in predicate$>$} \\
&|& \textrm{$<$null predicate$>$} \\

% \textrm{$<$exists predicate$>$} &::=& \textrm{[ NOT ] EXISTS ( $<$query expression$>$ )} \\
% 
% \textrm{$<$query expression$>$} &::=& \textrm{SELECT *}	\\
% % && \textrm{FROM $R_{1}$ $x_{1}$,$\cdots$, $R_{n}$ $x_{n}$ }	\\
% && \textrm{FROM $<$table list$>$}	\\
% && \textrm{WHERE $<$search condition$>$} \\
% 
% \textrm{$<$table list$>$} &::=& \textrm{$<$table name$>$} ~ \textrm{$<$tuple name$>$} \\
% && \textrm{$<$table list$>$} , \textrm{$<$table name$>$} ~ \textrm{$<$tuple name$>$} \\
% 
\textrm{$<$comparison predicate$>$} &::=& \textrm{$<$expression$_{1}>$} ~ \textrm{$<$comp op$>$} ~ \textrm{$<$expression$_{2}>$} \\
\textrm{$<$comp op$>$} &::=&  =	
~|~	<>
~|~	<
~|~	\leq
~|~	>
~|~	\geq \\

\textrm{$<$expression$>$} &::=& \textrm{$<$term$>$} \\
&|& \textrm{$<$expression$>$} ~ \left\{ + ~|~ - \right\} ~ \textrm{$<$term$>$} \\

\textrm{$<$term$>$} &::=& \textrm{$<$factor$>$} \\
&|&	 \textrm{$<$term$>$} ~ \left\{ * ~|~ / \right\} ~ \textrm{$<$factor$>$} \\
  
\textrm{$<$factor$>$} &::=& \left( \textrm{$<$expression$>$} \right) \\
&|& \left[ + ~|~ - \right] \textrm{\textit{$<$const$>$}} \\
&|& \left[ + ~|~ - \right] \textrm{$<$column$>$} \\

\textrm{$<$column$>$} &::=& \textrm{\textit{$<$tuple name$>$}.\textit{$<$attribute name$>$}} \\

\textrm{$<$between predicate$>$} &::=& \mbox{$<$expression$>$ [ NOT ]} \\
&& \mbox{BETWEEN \textit{$<$const$_{1}>$} AND \textit{$<$const$_{2}>$}} \\

\textrm{$<$in predicate$>$}	&::=& \textrm{$<$expression$>$} \textrm{[ NOT ] IN ( $<$in value list$>$ )} \\
\textrm{$<$in value list$>$} &::=& \textrm{\textit{$<$const$>$}} \\
&|& \textrm{$<$in value list$>$} , \textrm{\textit{$<$const$>$}} \\

\textrm{$<$null predicate$>$} &::=& \textrm{$<$column$>$} \textrm{ IS [ NOT ] NULL} \\

\edm

The general form of the target Why3ML code is:
\begin{center}
predicate $<$assertion name$>$ $<$parameters$>$ = \\
$<$logical formula$>$
\end{center}

We define the function $\mathcal{T}$ as the translational function mapping a SQL assertion phrase into a logical formula.
\bdm

\begin{array}{r}
\mathcal{T} [ \textrm{CREATE ASSERTION $<$assertion name$>$} \\
\textrm{CHECK $<$exists predicate$>$} ]
\end{array}
& \leadsto & \mathcal{T} [ \textrm{$<$exists predicate$>$} ]
\\
\begin{array}[t]{rl}
  \mathcal{T} [ \textrm{EXISTS (} & \textrm{SELECT *} \\
  & \textrm{FROM $R_{1}$ $x_{1}$,$\cdots$, $R_{n}$ $x_{n}$}	\\
  & \textrm{WHERE $<$search condition$>$ )} ]
\end{array}
&\leadsto&
\begin{array}[t]{l}
  \textrm{exists } x_{1}, \ldots , x_{n}. \\
  x_{1} \in R_{1} \land \dots \land x_{n} \in R_{n} \\
  \land \mathcal{T} [ \textrm{$<$search condition$>$} ]
\end{array}
\\
\begin{array}[t]{rl}
  \mathcal{T} [ \textrm{NOT EXISTS (} & \textrm{SELECT *} \\
  & \textrm{FROM $R_{1}$ $x_{1}$,$\cdots$, $R_{n}$ $x_{n}$}	\\
  & \textrm{WHERE $<$search condition$>$ )} ]
\end{array}
&\leadsto&
\begin{array}[t]{l}
  \textrm{not (exists } x_{1}, \ldots , x_{n}. \\
  x_{1} \in R_{1} \land \dots \land x_{n} \in R_{n} \\
  \land \mathcal{T} [ \textrm{$<$search condition$>$} ] )
\end{array}
\edm
\bdm
\mathcal{T} [ \textrm{$<$search condition$>$ OR }  \textrm{$<$boolean term$>$} ] &\leadsto& \mathcal{T} [ \textrm{$<$search condition$>$} ] \lor \mathcal{T} [ \textrm{$<$boolean term$>$} ] 
\\
\mathcal{T} [ \textrm{$<$boolean term$>$ AND }  \textrm{$<$boolean factor$>$} ] &\leadsto& \mathcal{T} [ \textrm{$<$boolean term$>$} ] \land \mathcal{T} [ \textrm{$<$boolean factor$>$} ] 
\\
\mathcal{T} [ \textrm{NOT ($<$predicate$>$)} ] &\leadsto&  \lnot ~ (\mathcal{T} [ \textrm{$<$predicate$>$} ] )
\edm
\bdm
\mathcal{T} [ \textrm{$<$expression$_{1}>$} ~ \textrm{$<$comp op$>$} ~ \textrm{$<$expression$_{2}>$} ] 
&\leadsto& 
\begin{array}[t]{l}
  \mathcal{T} [ \textrm{$<$expression$_{1}>$} ] \\  \mathcal{T} [ \textrm{$<$comp op$>$} ] ~ \mathcal{T} [ \textrm{$<$expression$_{2}>$} ]
\end{array}
\\
\textrm{$<$comp op$>$} &::=&  =	
~|~	<>
~|~	<
~|~	\leq
~|~	>
~|~	\geq 
\\
\mathcal{T} [ \textrm{$<$comp op$>$} ] &\leadsto& \textrm{$<$comp op$>$}
\\
\mathcal{T} [ \textrm{$<$expression$_{1}>$} ~ \textrm{$<$numerical op$>$} ~ \textrm{$<$expression$_{2}>$} ] 
&\leadsto& % \\ &&
\begin{array}[t]{l}
  \mathcal{T} [ \textrm{$<$expression$_{1}>$} ] \\  \mathcal{T} [ \textrm{$<$numerical op$>$} ] ~ \mathcal{T} [ \textrm{$<$expression$_{2}>$} ]
\end{array}
\\
\textrm{$<$numerical op$>$} &::=& + ~|~ - ~|~ \times ~|~ /
\\
\mathcal{T} [ \textrm{$<$numerical op$>$} ] &\leadsto& \textrm{$<$numerical op$>$}
\\
\mathcal{T} [ \textrm{$<$const$>$} ] &\leadsto& \textrm{$<$const$>$}
\\
\mathcal{T} [ x.a ] &\leadsto& x.a 
\edm
\bdm
\begin{array}[t]{l}
\mathcal{T} [ \mbox{$<$expression$>$ BETWEEN} \\ \mbox{$<$const$_{1}>$ AND $<$const$_{2}>$} ]
\end{array}
&\leadsto& 
\begin{array}[t]{l}
% (
  \left( \mathcal{T} [ \textrm{$<$expression$>$} ] ~\geq~ \textrm{$<$const$_{1}>$} \right)	
  \\	\land 
  \left( \mathcal{T} [ \textrm{$<$expression$>$} ] ~\leq~ \textrm{$<$const$_{2}>$} \right)
%  )
\end{array}
\\
\begin{array}[t]{l}
\mathcal{T} [ \mbox{$<$expression$>$ NOT BETWEEN} \\ \mbox{$<$const$_{1}>$ AND $<$const$_{2}>$} ] 
\end{array}
&\leadsto& 
\begin{array}[t]{l}
  \left( \mathcal{T} [ \textrm{$<$expression$>$} ] ~<~ \textrm{$<$const$_{1}>$} \right)	
  \\ \land 
  \left( \mathcal{T} [ \textrm{$<$expression$>$} ] ~>~ \textrm{$<$const$_{2}>$} \right)
\end{array}
\edm
\bdm
\begin{array}[t]{l}
  \mathcal{T} [ \textrm{$<$expression$>$ IN}	\\	\textrm{ ( $<$const$_{1}>$, \ldots, $<$const$_{m}>$ ) } ]
\end{array}
&\leadsto&  
\begin{array}[t]{l}
  \textrm{mem } \mathcal{T}	[ \textrm{$<$expression$>$} ] \\
  \textrm{(Cons $<$const$_{1}>$ (Cons \ldots (Cons $<$const$_{m}>$ Nil) \ldots ))}
\end{array}
\\
\begin{array}[t]{l}
  \mathcal{T} [ \textrm{$<$expression$>$ NOT IN}	\\	\textrm{ ( $<$const$_{1}>$, \ldots, $<$const$_{m}>$ ) } ]
\end{array}
&\leadsto&  
\begin{array}[t]{l}
  \textrm{not (mem } \mathcal{T}	[ \textrm{$<$expression$>$} ] \\
  \textrm{(Cons $<$const$_{1}>$ (Cons \ldots (Cons $<$const$_{m}>$ Nil) \ldots )))}
\end{array}
\edm
\bdm
\begin{array}{l}
\mathcal{T} [ x.a \textrm{ IS NULL} ]
\end{array}
&\leadsto&  x.a = \textrm{NULL} 
\\
\begin{array}{l}
\mathcal{T} [ x.a \textrm{ IS NOT NULL} ]
\end{array}
&\leadsto&  x.a \neq \textrm{NULL} 
\\
\mbox{Let $exp$ be a source language phrase, then:}
\\
\mathcal{T} [ ( exp ) ] &\leadsto& ( \mathcal{T} [ exp ] )

\edm

\section{Parser for the SQL INSERT statement}
The grammar of SQL insert statement is:
\bdm
\textrm{$<$insert statement$>$} &::=& \textrm{INSERT INTO \textit{$<$table name$>$} VALUES ( $<$column value list$>$ )} \\
&|& \textrm{INSERT INTO \textit{$<$table name$>$} ( $<$column name list$>$ )} \\
&& \textrm{VALUES ( $<$column value list$>$ )} \\
\textrm{$<$column name list$>$} &::=& \textrm{\textit{$<$column name$>$}} \\
&|& \textrm{$<$column name list$>$} , \textrm{\textit{$<$column name$>$}} \\
\textrm{$<$column value list$>$} &::=& \textrm{$<$column value$>$} \\
&|& \textrm{$<$column value list$>$} , \textrm{\textit{$<$column value$>$}} \\

\edm
The general form of the target Why3ML code is:
\bdm
\textrm{$<$insert function$>$} &::=&
\textrm{let \textit{$<$fun name$>$} \textit{$<$fun parameters$>$}} = \\
&& \textrm{\{ $<$precondition$>$ \}} \\
&& \textrm{\textit{$<$target table$>$} ++ $<$new tuple$>$} \\
&& \textrm{\{ $<$postcondition$>$ \}} \\
\textrm{$<$precondition$>$} &::=&
\textrm{\textit{$<$assertion name$>$} \textit{$<$assertion arguments$>$}} \\
&|& \textrm{$<$precondition$>$} \land \textrm{\textit{$<$assertion name$>$} \textit{$<$assertion arguments$>$}} \\
\textrm{$<$new tuple$>$} &::=& \textrm{( Cons \{$|<$new column list$>|$\} Nil )} \\
\textrm{$<$new column list$>$} &::=& 
\textrm{\textit{$<$column name$>$} = \textit{$<$column value$>$}} \\
&|& \textrm{$<$new column list$>$; \textit{$<$column name$>$} = \textit{$<$column value$>$}} 
\edm
The grammar of $<$postcondition$>$ is the same as that of $<$precondition$>$ except that all occurrences of $<$target table$>$ in the $<$assertion arguments$>$ are replaced by ``result''.

\section{Parser for the SQL DELETE statement}
The grammar of SQL delete statement is:
\bdm
\textrm{$<$delete statement$>$} &::=& \textrm{DELETE FROM \textit{$<$target table name$>$}} \\
&& \textrm{[ USING $<$table reference list$>$ ]} \\
&& \textrm{[ WHERE $<$search condition$>$ ]} \\
\textrm{$<$table reference list$>$} &::=& \textrm{\textit{$<$table name$>$}} \\
&|& \textrm{$<$table reference list$>$} , \textrm{\textit{$<$table name$>$}} \\

\textrm{$<$search condition$>$} &::=& \textrm{$<$boolean term$>$} \\
&|& \textrm{$<$search condition$>$ OR $<$boolean term$>$} \\

\textrm{$<$boolean term$>$} &::=& \textrm{$<$boolean factor$>$} \\
&|& \textrm{$<$boolean term$>$ AND $<$boolean factor$>$} \\

\textrm{$<$boolean factor$>$} &::=& \textrm{$<$predicate$>$} \\
&|& \textrm{[ NOT ] ( $<$search condition$>$ )} \\

\textrm{$<$predicate$>$} &::=& \textrm{$<$comparison predicate$>$} \\
&|& \textrm{$<$between predicate$>$} \\
&|& \textrm{$<$in predicate$>$} \\
&|& \textrm{$<$null predicate$>$} \\
\edm
The left parts are the same as those in the SQL assertion, so they are omitted in this manual. 

If $<$search condition$>$ is not specified in the delete statement, then the general form of the target Why3ML code is:
\bdm
\textrm{$<$delete function$>$} &::=&
\textrm{let rec \textit{$<$fun name$>$} $<$fun parameters$>$} = \\
&& \textrm{\{ true \}} \\
&& \textrm{match \textit{$<$target table$>$} with} \\
&& \textrm{$|$ Nil $\rightarrow$ Nil} \\
&& \textrm{$|$ Cons \{$|$ $<$tuple exp$>$ $|$\} \textit{$<$left table$>$}} \rightarrow \\
&& \textrm{( \textit{$<$fun name$>$} \textit{$<$fun arguments$>$} )} \\
&& \textrm{end} \\
&& \textrm{\{ $<$postcondition$>$ \}} \\
\textrm{$<$tuple exp$>$} &::=&
\textrm{\textit{$<$column name$>$} = $<$column value string$>$} \\
&|& \textrm{$<$tuple exp$>$; \textit{$<$column name$>$} = $<$column value string$>$} \\ 
\textrm{$<$column value string$>$} &::=& \textrm{\textit{$<$table name$>$}\_\textit{$<$column name$>$}\_value} \\
\textrm{$<$postcondition$>$} &::=& \textrm{$<$condition$>$} \rightarrow \textrm{$<$consequence$>$} \\
\textrm{$<$condition$>$} &::=& 
\textrm{\textit{$<$assertion name$>$} \textit{$<$assertion arguments$>$}} \\
&|& \textrm{$<$condition$>$} \land \textrm{\textit{$<$assertion name$>$} \textit{$<$assertion arguments$>$}} \\
\edm
The grammar of $<$fun arguments$>$ is the same as that of $<$fun parameters$>$ except that all occurrences of $<$target table$>$ are replaced by $<$left table$>$.
The grammar of $<$consequence$>$ is the same as that of $<$condition$>$ except that all occurrences of $<$target table$>$ in the $<$assertion arguments$>$ are replaced by ``result''.

If $<$search condition$>$ is specified and there is only one table in the delete statement, then the general form of the target Why3ML code is:
\bdm
\textrm{$<$delete function$>$} &::=&
\textrm{let rec $<$fun name$>$ $<$fun parameters$>$} = \\
&& \textrm{\{ true \}} \\
&& \textrm{match $<$target table$>$ with} \\
&& \textrm{$|$ Nil $\rightarrow$ Nil} \\
&& \textrm{$|$ Cons \{$|$ $<$tuple exp$>$ $|$\} $<$left table$>$} \rightarrow \\
&& \textrm{if $<$search condition$>$ then ( $<$fun name$>$ $<$fun arguments$>$ )} \\
&& \textrm{else Cons \{$|$ $<$tuple exp$>$ $|$\} ( $<$fun name$>$ $<$fun arguments$>$ )} \\
&& \textrm{end} \\
&& \textrm{\{ $<$postcondition$>$ \}} \\

\edm

If $<$search condition$>$ is specified and more than one tables are involved in the delete statement, then we generate a predicate, a set of iteration functions and a delete function.

The predicate is used to represent the $<$search condition$>$, which will be used in the postcondition part of the iteration functions and the delete function. The general form of the predicate is:
\bdm
\textrm{$<$sc predicate$>$} &::=&
\textrm{predicate $<$predicate name$>$ $<$predicate parameters$>$ =} \\
&& \textrm{$<$search condition$>$}
\edm
The iteration functions are used to obtain the required column values from tables other from the target table. The general form of the iteration function is:
\bdm
\textrm{$<$iter function$>$} &::=&
\textrm{let rec \textit{$<$fun name$>$} $<$fun parameters$>$} = \\
&& \textrm{\{ $<$precondition$>$ \}} \\
&& \textrm{match $<$iter table$>$ with} \\
&& \textrm{$|$ Nil $\rightarrow$ False} \\
&& \textrm{$|$ Cons \{$|$ $<$tuple exp$>$ $|$\} $<$left table$>$} \rightarrow \\
&& \textrm{if $<$check condition$>$ then True} \\
&& \textrm{else ( $<$fun name$>$ $<$fun arguments$>$ )} \\
&& \textrm{end} \\
&& \textrm{\{ $<$postcondition$>$ \}} \\
\textrm{$<$precondition$>$} &::=&
\textrm{\textit{$<$assertion name$>$} \textit{$<$assertion arguments$>$}} \\
&|& \textrm{$<$precondition$>$} \land \textrm{\textit{$<$assertion name$>$} \textit{$<$assertion arguments$>$}} \\
\edm
Let $ITL$ be the list of tables that have been already iterated, then $\forall arg \in \textrm{$<$assertion arguments$>$}, arg \in ITL$.
\bdm
\textrm{$<$postcondition$>$} &::=&
\textrm{$<$precondition$>$} \land \textrm{( result = True $\rightarrow$ $<$exists statement$>$ )} \\
\textrm{$<$exists statement$>$} &::=&
\textrm{exists $<$tuple$>$: $<$tuple type$>$.} \\
&& \textrm{mem $<$tuple$>$ $<$iter table$>$ $\land$ ( $<$sc predicate$>$ $<$predicate arguments$>$ )}
\edm

The delete function is the function that will delete tuples from the target table. The general form of the delete function is:
\bdm
\textrm{$<$delete function$>$} &::=&
\textrm{let rec $<$fun name$>$ $<$fun parameters$>$} = \\
&& \textrm{\{ $<$precondition$>$ \}} \\
&& \textrm{match $<$target table$>$ with} \\
&& \textrm{$|$ Nil $\rightarrow$ False} \\
&& \textrm{$|$ Cons \{$|$ $<$tuple exp$>$ $|$\} $<$left table$>$} \rightarrow \\
&& \textrm{if $<$check condition$>$ then ( $<$fun name$>$ $<$fun arguments$>$ )} \\
&& \textrm{else Cons \{$|$ $<$tuple exp$>$ $|$\} ( $<$fun name$>$ $<$fun arguments$>$ )} \\
&& \textrm{end} \\
&& \textrm{\{ $<$postcondition$>$ \}} \\


\edm


\section{Parser for the SQL UPDATE statement}
The grammar of SQL update statement is:
\bdm
\textrm{$<$update statement$>$} &::=& \textrm{UPDATE \textit{$<$target table name$>$}} \\
&& \textrm{SET $<$set clause list$>$} \\
&& \textrm{[ FROM $<$table reference list$>$ ]} \\
&& \textrm{[ WHERE $<$search condition$>$ ]} \\
\textrm{$<$set clause list$>$} &::=& \textrm{$<$set clause$>$} \\
&& \textrm{$<$set clause list$>$} , \textrm{$<$set clause$>$} \\
\textrm{$<$set clause$>$} &::=& \textrm{$<$set column$>$} = \textrm{\textit{$<$const$>$}} \\
\textrm{$<$set column$>$} &::=& \textrm{\textit{$<$table name$>$}.\textit{$<$attribute name$>$}} \\
\edm
The left parts are the same as those in the SQL delete statement grammar, so they are omitted in this manual. 

If $<$search condition$>$ is not specified in the update statement, then the general form of the target Why3ML code is:
\bdm
\textrm{$<$update function$>$} &::=&
\textrm{let rec $<$fun name$>$ $<$fun parameters$>$} = \\
&& \textrm{\{ true \}} \\
&& \textrm{match $<$target table$>$ with} \\
&& \textrm{$|$ Nil $\rightarrow$ Nil} \\
&& \textrm{$|$ Cons \{$|$ $<$old tuple exp$>$ $|$\} $<$left table$>$} \rightarrow \\
&& \textrm{Cons \{$|$ $<$new tuple exp$>$ $|$\} ( $<$fun name$>$ $<$fun arguments$>$ )} \\
&& \textrm{end} \\
&& \textrm{\{ $<$postcondition$>$ \}} \\
\textrm{$<$tuple exp$>$} &::=&
\textrm{$<$column name$>$ = $<$column value string$>$} \\
&|& \textrm{$<$tuple exp$>$; $<$column name$>$ = $<$column value string$>$} \\ 
\textrm{$<$column value string$>$} &::=& \textrm{$<$table$>_<$column$>_<$value$>$} \\
\textrm{$<$postcondition$>$} &::=& \textrm{$<$condition$>$} \rightarrow \textrm{$<$consequence$>$} \\
\textrm{$<$condition$>$} &::=& 
\textrm{\textit{$<$assertion name$>$} \textit{$<$assertion arguments$>$}} \\
&|& \textrm{$<$condition$>$} \land \textrm{\textit{$<$assertion name$>$} \textit{$<$assertion arguments$>$}} \\
\edm
The grammar of $<$fun arguments$>$ is the same as that of $<$fun parameters$>$ except that all occurrences of $<$target table$>$ are replaced by $<$left table$>$.
The grammar of $<$consequence$>$ is the same as that of $<$condition$>$ except that all occurrences of $<$target table$>$ in the $<$assertion arguments$>$ are replaced by ``result''.

If $<$search condition$>$ is specified and there is only one table in the update statement, then the general form of the target Why3ML code is:
\bdm
\textrm{$<$update function$>$} &::=&
\textrm{let rec $<$fun name$>$ $<$fun parameters$>$} = \\
&& \textrm{\{ true \}} \\
&& \textrm{match $<$target table$>$ with} \\
&& \textrm{$|$ Nil $\rightarrow$ Nil} \\
&& \textrm{$|$ Cons \{$|$ $<$old tuple exp$>$ $|$\} $<$left table$>$} \rightarrow \\
&& \textrm{if $<$search condition$>$} \\
&& \textrm{then Cons \{$|$ $<$new tuple exp$>$ $|$\} ( $<$fun name$>$ $<$fun arguments$>$ )} \\
&& \textrm{else Cons \{$|$ $<$old tuple exp$>$ $|$\} ( $<$fun name$>$ $<$fun arguments$>$ )} \\
&& \textrm{end} \\
&& \textrm{\{ $<$postcondition$>$ \}} \\

\edm

If $<$search condition$>$ is specified and more than one tables are involved in the update statement, then we generate a predicate, a set of iteration functions and a update function.

The predicate is used to represent the $<$search condition$>$, which will be used in the postcondition part of the iteration functions and the update function. The general form of the predicate is:
\bdm
\textrm{$<$sc predicate$>$} &::=&
\textrm{predicate $<$predicate name$>$ $<$predicate parameters$>$ =} \\
&& \textrm{$<$search condition$>$}
\edm
The iteration functions are used to obtain the required column values from tables other from the target table. The general form of the iteration function is:
\bdm
\textrm{$<$iter function$>$} &::=&
\textrm{let rec $<$fun name$>$ $<$fun parameters$>$} = \\
&& \textrm{\{ $<$precondition$>$ \}} \\
&& \textrm{match $<$iter table$>$ with} \\
&& \textrm{$|$ Nil $\rightarrow$ False} \\
&& \textrm{$|$ Cons \{$|$ $<$tuple exp$>$ $|$\} $<$left table$>$} \rightarrow \\
&& \textrm{if $<$check condition$>$ then True} \\
&& \textrm{else ( $<$fun name$>$ $<$fun arguments$>$ )} \\
&& \textrm{end} \\
&& \textrm{\{ $<$postcondition$>$ \}} \\
\textrm{$<$precondition$>$} &::=&
\textrm{\textit{$<$assertion name$>$} \textit{$<$assertion arguments$>$}} \\
&|& \textrm{$<$precondition$>$} \land \textrm{\textit{$<$assertion name$>$} \textit{$<$assertion arguments$>$}} \\
\edm
Let $ITL$ be the list of tables that have been already iterated, then $\forall arg \in \textrm{$<$assertion arguments$>$}, arg \in ITL$.
\bdm
\textrm{$<$postcondition$>$} &::=&
\textrm{$<$precondition$>$} \land \textrm{( result = True $\rightarrow$ $<$exists statement$>$ )} \\
\textrm{$<$exists statement$>$} &::=&
\textrm{exists $<$tuple$>$: $<$tuple type$>$.} \\
&& \textrm{mem $<$tuple$>$ $<$iter table$>$ $\land$ ( $<$sc predicate$>$ $<$predicate arguments$>$ )}
\edm

The update function is the function that will update tuples from the target table. The general form of the update function is:
\bdm
\textrm{$<$update function$>$} &::=&
\textrm{let rec $<$fun name$>$ $<$fun parameters$>$} = \\
&& \textrm{\{ $<$precondition$>$ \}} \\
&& \textrm{match $<$target table$>$ with} \\
&& \textrm{$|$ Nil $\rightarrow$ False} \\
&& \textrm{$|$ Cons \{$|$ $<$old tuple exp$>$ $|$\} $<$left table$>$} \rightarrow \\
&& \textrm{if $<$check condition$>$} \\
&& \textrm{then Cons \{$|$ $<$new tuple exp$>$ $|$\} ( $<$fun name$>$ $<$fun arguments$>$ )} \\
&& \textrm{else Cons \{$|$ $<$new tuple exp$>$ $|$\} ( $<$fun name$>$ $<$fun arguments$>$ )} \\
&& \textrm{end} \\
&& \textrm{\{ $<$postcondition$>$ \}} \\


\edm


\end{document}


